\documentclass{article}
\usepackage{enumitem}
\usepackage{amsmath}

\begin{document}

\noindent
CSCI 6470\\
Homework 6\\
Jiakuan Li\\

\noindent
1.
\begin{enumerate}[label={}]

\item
If we can solve the Clique problem in polynomial time, so the Max Clique problem.\\
The algorithm is easy, we just need to keep iterate the algorithm of that solve the Clique problem.\\
Let $k=|G|$ at the very beginning. Keep execute the Clique problem. For each iteration, let $k = k -1$. Stop if we have a "YES" return by the Clique algorithm 

\end{enumerate}

\noindent
2.
\begin{enumerate}[label={}]
\item
The previous algorithm works if the symbol in Clique problem change to "=". However, the algorithm does not work if the symbols is "$\leq$". Since both "$\geq$" and "=" sets the upper limit for Clique, so, it guarantee that $|C|$ is at least $k$. The "$\leq$" symbols sets the lower limit for the Clique, it guarantee that $|C|$ is at most $k$. So, we could not solve the Max Clique by solving the Clique Problem. Therefore, if the symbol in Clique problem changes to "$\leq$", and we could solve the Clique problem in polynomial time, we could not guarantee to solve the Max-Clique problem in polynomial time.
\end{enumerate}

\noindent
3.
\begin{enumerate}[label={}]
\item
Verifier:\\
Take a graph $G<V,E>$ and integer $k$.And a set $S$.\\
Check if $|S| \geq k$, return "NO" if false.\\
Check if $(u,v)\in E$ for every $(u,v)$ pair in set $S$.\\
This verifier runs in $O(n^2)$.\\
Prove complete.
\end{enumerate}
\noindent
4.
\begin{enumerate}[label={}]
\item
TODO
\end{enumerate}

\noindent
5.
\begin{enumerate}[label={}]
\item
Given an undirected graph $G=<V,E>$\\
Define a complement of $G$ as $\overline{G}=<V, \overline{E}>$.\\
$\overline{G}$ has the same set of vertices as $G$, but $\overline{E}\cap E = \emptyset$.\\
If there exists a Clique $V'$ in $\overline{G}$, the $V-V'$ is a vertex cover in G.\\
\\
The reduction as follow:\\
1. Takes $G=<V,E>$ and integer $k$.\\
2. Generate $\overline{G}$.\\
3. solve Clique($\overline{G}$, $k$).\\
4. If there's a solution, return "YES".
\end{enumerate}

\noindent
6.
\begin{enumerate}[label={}]
\item
Since $L_1 \le_P L_2$, so $L_2$ is at least as hard as $L_1$. In another word, $L_1$ will not be harder than $L_2$. Since $L_2$ admits an algorithm that runs in $O(n^{O(\log n)})$, so for an easier problem, $L_1$ should not run slower than $O(n^{O(\log n)})$. The upper bound for $L_1$ is $O(n^{O(\log n)})$. Therefore, if $L_1 \le_P L_2$, and $L_2$ admits an algorithm runs in $O(n^{O(\log n)})$, so do $L_1$.
\end{enumerate}

\noindent
7.
\begin{enumerate}[label={}]
\item
Since $L' \le_P L$, so $L'$ is polynomial time reduction to $L$. This means $L$ is at least as hard as $L'$. So if $L\in NP-Complete$, then $L'\in NP-Complete$ as well. 
\end{enumerate}
\end{document}